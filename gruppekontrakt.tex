% Dokumentklassen s?tes til memoir.
% Manual: http://ctan.org/tex-archive/macros/latex/contrib/memoir/memman.pdf
\documentclass[a4paper,oneside,article]{memoir}

\usepackage{cmap}

% Danske udtryk (fx figur og tabel) samt dansk orddeling og fonte med
% danske tegn. Hvis LaTeX brokker sig over ? ? og ?skal du udskifte
% "utf8" med "latin1" eller "applemac".
\usepackage[utf8]{inputenc}
\usepackage[danish]{babel} % 'danish' erstattes med 'english', hvis der skrives p?engelsk
\usepackage[T1]{fontenc}


%\usepackage{hyperref}
\usepackage[pdftex, pdfborderstyle={/S/U/W 0}]{hyperref} % this disables the boxes around links

\usepackage{hyperref} %for clickable urls
\usepackage{easylist}
\usepackage{float}
\usepackage{lastpage}

\newcommand{\HRule}{\rule{\linewidth}{0.5mm}}

\newcommand{\navn}{Mark Moore }
\newcommand{\fuldenavn}{Mark Robert Nygaard Moore }
\newcommand{\studnr}{20061263 }
\newcommand{\hold}{IT3 }
\newcommand{\email}{mark@moore.dk }
\newcommand{\mobil}{24404972 }


\newcommand{\fag}{Gruppekontrakt }
\newcommand{\titel}{For HTTP/1.1 418 }
%Aflevering nummer:
\setcounter{chapter}{1}

% Section numbering depth
%\setcounter{secnumdepth}{3}
\setsecnumdepth{subsection}


%make section number appear AFTER section name:
%\usepackage[explicit]{titlesec}
%\titleformat{\section}{\normalfont\Large\bfseries}{}{0em}{#1\ \thesection}

%TOC depth
\setcounter{tocdepth}{2}

% Inds?telse af grafik.
\usepackage{graphicx}
\usepackage{wrapfig}

% Header
\setheaderspaces{*}{5\onelineskip}{*}
\makepagestyle{sitin}

% Margin
\setlrmarginsandblock{2cm}{2cm}{*}
\checkandfixthelayout
\newcommand{\pagewidth}{\textwidth}  % 17 cm
\newcommand{\pageheight}{\textheight} % 20 cm

\makeoddhead{sitin}
	%Left
	{
		\includegraphics[height=4\onelineskip]{LargeAULogo.png}
	}
	%Center
	{
		\fag \\
		\titel \\
	}
	%Right
	{
		\includegraphics[height=4\onelineskip]{kimJongTeapot.jpg}
	}
\makeoddfoot{sitin}{}{}{Side \thepage\space af \pageref{LastPage}}

\makeheadrule{sitin}{\textwidth}{.4pt}
%\makefootrule{sitin}{\textwidth}{.4pt}{0.1cm}LargeAULogo

\pagestyle{sitin}


%-------- Skriv afleveringdfrist
\date {04. februar 2015}


\begin{document}
%\begin{titlingpage} \maketitle \end{titlingpage}
%\begin{titlingpage} \tableofcontents \end{titlingpage}
%----------------------Her skriver du din opgave-------------------%

\section{Indledning}
Dette er gruppekontrakten for gruppen "HTTP/1.1 418".
Kontrakten beskriver gruppens - i enighed - vedtagne regler og retningslinier for medlemmernes faglige aktiviteter under IT bachelor studiet på Aarhus Universitet.
NOTE: Dette er en kladde. Intet er vedtaget.\\
Gruppen består af:
\subsection{Gruppemedlemmer}
\begin{tabular}{ l | l | l | l }
    Navn & Studie\# $\triangle$ & Mobil\# & Email \\ \hline
    Mark Robert Nygaard Moore & 20061263 & 24 40 49 72 & mark@moore.dk \\ \hline
    Sebastian Skjold Højsted & 201407642 & 31 24 01 33 & sthazeb@gmail.com \\ \hline
	Rasmus Mølgaard Hansen & 201408297 & 23 96 29 77 & ramusmh@gmail.com \\ \hline
    Silas Roswall & 201408919 & 25 88 45 86 & silasrw@gmail.com
\end{tabular}
\label{subsec:medlemmer}
\section{Vedtægter}
\subsection{Mødetider}
Vi møder til den (af universitetet) skemalagte tid med akademisk kvarter, hvor det gør sig gældende.
Bliver man mere end 15 minutter forsinket meddeles dette til mindst ét andet gruppemedlem. (fx via sms eller facebook)
Modtager man sms om forsinkelse mens man selv er forsinket, gives beskeden videre.
\subsection{Ugentligt Møde}
Gruppen holder møde en gang om ugen, primært med henblik på at få overblik over uddelegerede, og kommende, opgaver.
\subsubsection{Udkast til dagsorden for ugentligt gruppemøde:}
\begin{enumerate}
\item Kaffe
\item Status på opgaver
\begin{enumerate}
\item Hvad er done?
\item Hvad mangler?
\item Hvad halter? - Hvorfor?
\item Uddelegering / redelegering af opgaver.
\item Ansvarlinge for aflevering af ugens opgaver.
\end{enumerate}
\item Planlægning af ugen:
\begin{enumerate}
\item Hvad gør vi hvornår? (og hvor mødes vi)?
\item Afbud?
\end{enumerate}
\item evt.
\end{enumerate}
\subsection{Gruppearbejde}
Gruppen arbejder så vidt muligt sammen i alle kurser da dette letter planlægningen af ugen.
Gruppen registrerer sig på de relevante kursussider, efter kursernes kravsatte omfang.
Det forventes at gruppemedlemmer påregner tid til gruppearbejde udenfor den skemalagte tid. Dette aftales løbende på det ugentlige møde, gerne i god tid.
Det forventes at gruppemedlemmer holder sig orienteret på facebook gruppen, og på fagportalerne.
\subsection{Afleveringer}
Der udpeges en "ansvarlig for aflevering"  pr. aflevering, som sørger for at en opgave afleveres/uploades til tiden.
\section{Ved Overtrædelse}
TODO: specificér nærmere\\
De øvrige gruppemedlemmer ser på de enkelte tilfælde af overtrædelser, og gransker alvorligheden af disse.\\
Straf kan udmåles fra kaffe til udelukkelse.
\section{Ændringer / overruling}
Bestemmelserne i gruppekontrakten kan til enhver tid ændres, såfremt gruppemedlemmerne bliver enige om en ændring.
\end{document}